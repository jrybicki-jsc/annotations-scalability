\documentclass[a4paper,10pt]{article}
\usepackage[utf8]{inputenc}
\usepackage[ngerman,english]{babel}
\usepackage{graphicx}

  
%opening
\title{neo4j performance and scalability evaluation}
\author{WP9.2}

\begin{document}

\maketitle

\begin{abstract}
this document evaluates scalability and performance of neo4j graph database in handling 
large annotation data sets. It compares the results with the currently used mongodb. 
\end{abstract}

\section{Introduction}
blah

\section{Methodology/Setup}
-annotaiton format
-programs
-deployement

\section{Results}
\begin{figure}
\centering
 \includegraphics[width=.7\textwidth]{fig/mongo-ret-scalability}
 \caption{"mongo retrieval times (db is growing \emph{rep} records in each round, \emph{rep} random records are retrieved)}
\end{figure}

\begin{figure}
\centering
 \includegraphics[width=.7\textwidth]{fig/neo-ret-scalability}
 \caption{neo4j retrieval scalability}
\end{figure}

\begin{figure}
\centering
 \includegraphics[width=.7\textwidth]{fig/creation}
 \caption{Comparison of creation scalability}
\end{figure}

\begin{figure}
\centering
 \includegraphics[width=.7\textwidth]{fig/neovsmongo}
 \caption{Scalability mongo vs. neo4j}
\end{figure}




\section{Summary}
Performance results should never be the final result of any project. They, rather, constitute
a starting point for further performance tuning. It is true for both mongo and neo4j. The database
management engines offer many possibles for parameter tuning. 

\end{document}

